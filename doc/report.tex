\documentclass[a4paper]{article}
%\usepackage{amsfonts}
\usepackage{amsmath}
%\usepackage{amsthm}
\usepackage[utf8]{inputenc}
%\usepackage{hyperref}
%\usepackage{booktabs}
%\usepackage{indentfirst}
\usepackage{graphicx}
%\usepackage{subfig}
\usepackage{minted}

\newminted{cpp}{
	fontsize=\footnotesize
}
\newminted{text}{
	fontsize=\footnotesize
}

\graphicspath{{fig/}}

\title{Box wrapping solver}
\author{Rodrigo Arias Mallo}
\date{\today}

% Useful vectorial and matrix notation
\newcommand*\mat[1]{ \begin{pmatrix} #1 \end{pmatrix}}
\newcommand*\arr[1]{ \begin{bmatrix} #1 \end{bmatrix}}
\newcommand*\V[1]{ \boldsymbol{#1}}

\begin{document}
\maketitle

\section{Constraint programming}

The solver \texttt{Gecode} was used to implement the variables and constraints, 
detailed in the \texttt{src/cp.cc} file.

\subsection{Variables}

Two integer variables $x_i$ and $y_i$ were used to set the \textsl{top-left} 
corner of each box $i$. The boolean array $rot_i$ determines whether the box $i$ 
is rotated (90 degrees) or not. In that case, the width and length are swapped 
for the box. The variable $L$ measures the length of the roll, and is used to 
set a constraint for each solution found, such that the next one needs to be at 
most $L - 1$.
%
\begin{cppcode}
L = IntVar(*this, min_L, max_L);
x = IntVarArray(*this, num_boxes, 0, W);
y = IntVarArray(*this, num_boxes, 0, max_L);
rot = BoolVarArray(*this, num_boxes, 0, 1);
\end{cppcode}
%
Some limits like $\min_L$ and $\max_L$ are computed from the total area of the 
boxes and the maximum length, when the boxes are vertical and stacked, to limit 
the range of the variables $L$ and $y$. The boxes are placed by using the 
coordinates of the corners. So a box at $(0,0)$ of width 2 and height 3 ends at 
$(2,3)$.

\subsection{Constraints}

We enforce that $L$ must be at least the lower part of each box, and $W$ the 
right side, so each box is placed inside the roll. If the box is rotated, then 
we swap $w_i$ and $h_i$.

\begin{cppcode}
rel(*this, (rot[i] && (L >= y[i] + w[i])) || (!rot[i] && (L >= y[i] + h[i])));
rel(*this, (rot[i] && (W >= x[i] + h[i])) || (!rot[i] && (W >= x[i] + w[i])));
\end{cppcode}
%
When a box is a square, we don't test for rotations.
%
\begin{cppcode}
if(w[i] == h[i]) rel(*this, rot[i] == 0);
\end{cppcode}
%
Finally, for each pair of boxes $i$ and $j$ with $i \neq j$, we add the 
following constraint.
%
\begin{cppcode}
rel(*this,
	(!rot[i] && (x[i] + w[i] <= x[j])) ||
	(!rot[j] && (x[j] + w[j] <= x[i])) ||
	(!rot[i] && (y[i] + h[i] <= y[j])) ||
	(!rot[j] && (y[j] + h[j] <= y[i])) ||

	( rot[i] && (x[i] + h[i] <= x[j])) ||
	( rot[j] && (x[j] + h[j] <= x[i])) ||
	( rot[i] && (y[i] + w[i] <= y[j])) ||
	( rot[j] && (y[j] + w[j] <= y[i]))
);
\end{cppcode}
%
The last four parts are equal to the first four, but when a box is rotated. Each 
of the four test that there is a direction in which the distance within the 
boxes is at least 0, in that case the box is allowed to be placed there.

\subsection{Branching policy}

In order to determine how the variables are explored, some policies were tested 
to find the fastest one. First we branch by $rot$ with the default value 0, (no 
rotation):
\begin{cppcode}
branch(*this, rot, BOOL_VAR_RND(r), BOOL_VAL_MIN());
\end{cppcode}
Then we branch $x$ and $y$, based on the policy selected. By default, the policy 
0 is:
\begin{cppcode}
branch(*this, x, INT_VAR_RND(r), INT_VAL_RND(r));
branch(*this, y, INT_VAR_RND(r), INT_VAL_RND(r));
\end{cppcode}
Whether the policy 1 is the following.
\begin{cppcode}
branch(*this, x, INT_VAR_SIZE_MIN(), INT_VAL_RND(r));
branch(*this, y, INT_VAR_SIZE_MIN(), INT_VAL_RND(r));
\end{cppcode}
Finally, we branch on $L$ using the minimum value posible:
\begin{cppcode}
branch(*this, L, INT_VAL_MIN());
\end{cppcode}
The branching policy can be selected from the command line, using the argument 
\texttt{-p <number>} using 0 or 1. Use \texttt{-h} for more information.

\subsection{Compilation}

In order to compile the code, check that the \texttt{PREFIX} variable in the 
\texttt{src/Makefile} is set to the correct value. In my system Gecode is 
installed in \texttt{/usr} but you may want to use \texttt{/usr/local}. Then, to 
compile, execute \texttt{make}.

\subsection{Running}

The solver \texttt{cp} reads from the standard input, and writes to the standard 
output. To read an instance you can use:
%
\begin{textcode}
% src/cp < in/bwp_10_4_1.in
3
9 2     9 2
3 0     5 2
6 0     8 2
0 0     2 2
\end{textcode}
%
Other options are available and are useful for debugging purposes, see 
\texttt{src/cp -h}. A \texttt{plot} program draws the box placement, but it 
needs to use an extended notation (\texttt{-e}). In the extended notation, the 
width of the paper roll is included in the first line, and the rotated boxes are 
indented.  Then it can be used as:
%
\begin{textcode}
% src/cp -e < in/bwp_10_4_1.in | src/plot
Plotting a roll of W = 10 and L = 3 with 4 boxes.
      0  1  2  3  4  5  6  7  8  9
   +------------------------------+
 0 |  4  4  4  2  2  2  3  3  3   |
 1 |  4  4  4  2  2  2  3  3  3   |
 2 |  4  4  4  2  2  2  3  3  3  1|
   +------------------------------+
\end{textcode}
%
Other options are documented in the builtin help:
\begin{textcode}
% src/cp -h
Constraint programming solver for the BWP.
Usage

	src/cp [options] < file.in > file.out

Options

  -h       Show this help

  -t <s>   Set the time limit to <s> seconds. Default 120.

  -c <c>   Set the number of threads to <c>. Default 1.

  -e       Use the extended notation, printing the width
           of the paper roll in the first line along the length.
           Needed for plotting but not compatible with checker.

  -v       Be verbose.

  -p <p>   Set the policy number <p> of the branching method for x and y.
           By default is 1, available policies:
             0 - INT_VAR_RND(r) and INT_VAL_RND(r)
             1 - INT_VAR_SIZE_MIN() and INT_VAL_RND(r)

\end{textcode}

\section{Linear programming}

The CPLEX solver with the C++ interface was used to implement the variables and 
constraints, detailed in the \texttt{src/lp.cc} file.

\subsection{Variables}

The same variables were used, $x_i$ and $y_i$ to set the \textsl{top-left} 
corner of the box $i$. As the $rot_i$ boolean array, and the length of the paper 
$L$.
%
\begin{cppcode}
IloIntVarArray x, y;
IloBoolVarArray rot;
IloIntVar L;
\end{cppcode}
%
Also the same limits where used, like $\min_L$ and $\max_L$ for $L$.
%
\begin{cppcode}
L = IloIntVar(env, min_L, max_L);
x = IloIntVarArray(env, num_boxes, 0, W);
y = IloIntVarArray(env, num_boxes, 0, max_L);
\end{cppcode}
%
\subsection{Constraints}

To enforce that the boxes should be inside the paper roll, we first compute the 
real width and height of each box, in case they are rotated:
%
\begin{cppcode}
IloExpr wi = (1 - rot[i]) * w[i] + rot[i] * h[i];
IloExpr hi = (1 - rot[i]) * h[i] + rot[i] * w[i];
\end{cppcode}
%
Then, we limit the right and bottom side, as the right and top are at least 0:
%
\begin{cppcode}
model.add(W >= x[i] + wi);
model.add(L >= y[i] + hi);
\end{cppcode}
%
Also, if a box is a square, we set $rot_i$ to 0:
%
\begin{cppcode}
if(w[i] == h[i]) model.add(rot[i] == 0);
\end{cppcode}
%
In order to avoid overlaping between two boxes $i$ and $j$, with $i \neq j$, we 
first compute the real size of each box:
%
\begin{cppcode}
IloExpr wi = (1 - rot[i]) * w[i] + rot[i] * h[i];
IloExpr hi = (1 - rot[i]) * h[i] + rot[i] * w[i];
IloExpr wj = (1 - rot[j]) * w[j] + rot[j] * h[j];
IloExpr hj = (1 - rot[j]) * h[j] + rot[j] * w[j];
\end{cppcode}
%
And then, we enforce that at least in one direction, the distance between the 
two boxes should be at least 0:
%
\begin{cppcode}
model.add(
		// x axis
		x[i] + wi <= x[j] || 
		x[j] + wj <= x[i] ||
		// y axis
		y[i] + hi <= y[j] ||
		y[j] + hj <= y[i]);
\end{cppcode}
%
In order to avoid duplicate placements of boxes with equal size, we assign an 
index to each box $i$ and $j$ with $i < j$ based on their position:
%
\begin{cppcode}
IloExpr pi = x[i] + y[i]*W;
IloExpr pj = x[j] + y[j]*W;
\end{cppcode}
%
And we only let the first box to have a smaller index:
%
\begin{cppcode}
model.add(pi <= pj);
\end{cppcode}
%
Finally, we add the objective to minimize $L$:
%
\begin{cppcode}
model.add(IloMinimize(env, L));
\end{cppcode}

\subsection{Compilation}

\subsection{Running}

For compiling, use \texttt{make} in the \texttt{src/} directory. The program 
\texttt{src/lp} reads the input file from the standard input.

To solve a specific instance use \texttt{src/lp < in/instance.in}. To run all 
the instances in \texttt{in/} use the script \texttt{./run.sh lp}.

\section{Automatic test}

In order to execute the solvers in all instances, a script was prepared for the 
task. It reads the technologies to be used from the parameters. For instance to 
execute with \texttt{lp} only, use \texttt{./run.sh lp}. Or with \texttt{cp} and 
\texttt{lp}, use \texttt{./run.sh cp lp}.

The script checks before that the solution was not already computed, otherwise 
skips the computation. If you want to try again some instances that were solved, 
remove the solutions first.

The instances are read from the \texttt{in/} directory, and the solutions and 
logs are placed in the \texttt{out/} directory. Check the log file for a 
specific instance to see the details of the solver. The output file is named 
\texttt{bwp\_$W$\_$N$\_$k$\_$t$.out} as expected, and the log file as 
\texttt{bwp\_$W$\_$N$\_$k$\_$t$.log}.

Once a solution is produced by the solver, the \texttt{checker} test if it was 
correct. In case it detects some error, the solution is saved in a temporal file 
appending \texttt{.tmp} to the file name. The output of the checker can be read 
in \texttt{bwp\_$W$\_$N$\_$k$\_$t$.chk}.

This behavior allows to determine which instances were incorrectly solved. If 
the program doesn't terminate after the time limit of 120 seconds, it is killed, 
and the solution is also considered incorrect, and \texttt{.tmp} is also 
appended.

\end{document}
